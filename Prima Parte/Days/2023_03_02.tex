\days{2 marzo 2023}
\section{Equazioni autonome in due dimensioni}
%% BEGIN Riduzione di un problema di Cauchy autonomo in due dimensioni 
\paragrafo{Orbite non singolari}{%
    Consideriamo l'equazione \[
        \begin{cases}
            x'=f_{1}(x,y)\\ 
            y' = f_{2}(x,y)  
        \end{cases}
    \]Supponiamo $ \bm{p}:\bm{f}(\bm{p})\neq \bm{0} $, $ \bm{p}=(p_1,p_2) $. Com è fatta l'orbita per $ \bm{p} $?

    Se $ \bm{f}(\bm{p})\neq \bm{0} $, allora almeno una delle sue componenti è non nulla. Supponiamo che $ f_{1}(\bm{p})\neq 0  $. 
    
    $\implies$ il vettore tangente all'orbita per $ \bm{p} $ non è verticale. 
    
    Consideriamo allora il sistema: \[
        \begin{cases}
            x'=f_{1}(x,y)\\ 
            y' = f_{2}(x,y)\\ 
            \left(x(0),y(0)\right)=\bm{p}
        \end{cases}
    \]e sia $ \bm{u} $ una soluzione massimale, tale che $ \bm{u}(0)=\bm{p} $. $ \bm{u}'(0) $ non è un vettore verticale 
    
    $\implies$ localmente posso esprimere la seconda componente dell'orbita per $ \bm{p} $ in funzione della prima. \[
        \exists!\, \left\{ \begin{aligned}
        \varphi: I_{p_1}  &\longrightarrow I_{p_2}  \\
        x &\longmapsto \varphi(x)
        \end{aligned}\right.
    \]e questa funzione descrive l'orbita \[
        \begin{cases}
            x=u_1(t)\\ 
            \begin{aligned}
                y=\varphi(x)&= u_2(t)\\ 
                & = \varphi\left(u_1(t)\right)
            \end{aligned}
        \end{cases}
    \]Si ha che \[
        f_2\left(\bm{u}(t)\right) = u_2'(t) = \varphi'\left(u_1(t)\right) \cdot u_1'(t)
    \]da cui $ \displaystyle f_2(x,y)=\varphi'(x)\,f_1(x,y) $ \[
        \varphi'(x)=\frac{f_2(x,y)}{f_1(x,y)} = \frac{f_2\left(x,\varphi(x)\right)}{f_1\left(x,\varphi(x)\right)}.
    \]

    Stiamo dicendo che se $ f_1(p)\neq 0 $ l'orbita che passa per $ \bm{p} $ la posso esprimere mediante una funzione $ y=\varphi(x) $ che soddisfa \[
        \begin{cases}
            \displaystyle\varphi'(x)=\frac{f_2\left(x,\varphi(x)\right)}{f_1\left(x,\varphi(x)\right)}\\[2ex]
            \varphi(p_1)=p_2
        \end{cases}
    \]ovvero un problema di Cauchy monodimensionale \emph{non autonomo}.
}{dafkjbnadlfjknasdlkjncdajklscnadlskjn}{}
\teorema{dlkjanbflkasjdbfnlakjnflkasjdnlkasdjnflaksdfjnlkjnlkjn}{
    Dato il problema di Cauchy \[
        \begin{cases}
            x'=f_1(x,y)\\ 
            y'=f_2(x,y)\\
            \left(x(0),y(0)\right)=\bm{p}
        \end{cases}\tag*{(PC)}
    \]con $ \bm{f}:\Omega \subseteq \R^{2}\to \R^{2}$ di classe $ C^{1} $. Supponiamo che $ f_1(\bm{p})\neq 0 $ (campo non verticale in $ \bm{p} $). Allora per degli opportuni intorni $ I_{p_1}, I_{p_2}$: 

    $ \varphi: I_{p_1}\longrightarrow I_{p_2}   $ è una rappresentazione locale dell'orbita della soluzione di (PC)

    $ \iff $ $ \varphi $ è soluzione di \[
        \begin{cases}
            \displaystyle\varphi'(x)=\frac{f_2\left(x,\varphi(x)\right)}{f_1\left(x,\varphi(x)\right)}\\[2ex]
            \varphi(p_1)=p_2
        \end{cases}\tag*{(PC)\textsubscript{$\varphi$}}
    \]
}
\dimostrazione{dlkjanbflkasjdbfnlakjnflkasjdnlkasdjnflaksdfjnlkjnlkjn}{
    \begin{itemize}
        \item[($\implies$)] Vedi: \framref{dafkjbnadlfjknasdlkjncdajklscnadlskjn}
        \item[($\impliedby$)] $ \varphi $ è soluzione di (PC)\textsubscript{$\varphi$}. Per ipotesi sappiamo che $ f_1\left(x,\varphi(x)\right) \neq 0$ in un opportuno intorno di $ p_1 $, $ I_{p_1}  $ (poiché $ f $ è $ C^{1} $). 
        
        Allora, $ \forall\, x \in I_{p_1}  $ sia\[
            \omega(x)=\int_{p_1}^{x}\frac{d\,\xi}{f_1\left(\xi, \varphi(\xi)\right)}.
        \]Poiché $ f_1 $ è continua e non nulla 
        
        $\implies$ $ \omega $ è derivabile e \[
            \begin{cases}
                \omega'(x)=\displaystyle\frac{1}{f_1\left(x,\varphi(x)\right)}\\[2ex]
                \omega(p_1)=0
            \end{cases}
        \]Poiché $ \omega' $ ha segno costante 
        
        $\implies$ $ \omega $ è strettamente monotona 
        
        $\implies$ $ \omega $ è invertibile. \[
             \begin{aligned}
                \omega: I_{p_1}\to I \ni 0 \quad\leadsto\quad\omega^{-1}=: v:I&\longrightarrow I_{p_1}\\ 
                v(0) &= p_1\\ 
                v'(t) &=\displaystyle \frac{1}{\omega'\left(v(t)\right)}  
            \end{aligned}
        \] 
        
        $\implies$ $ \displaystyle v'(t) = f_1\left(v(t), \varphi\left(v(t)\right)\right) $. 

        Se chiamo \[
            \bm{u}(t)\coloneqq\begin{pmatrix}
                v(t)\\\varphi\left(v(t)\right)
            \end{pmatrix}
        \]ho che \[
            \begin{cases}
                u_1'(t)=v'(t)=  f_1\left(v(t), \varphi\left(v(t)\right)\right) = f_1\left(u_1(t),u_2(t)\right)\\[2ex]
                u_2'(t)=\varphi'\left(v(t)\right)\,v'(t) = \displaystyle\frac{f_2}{f_1} f_1 = f_2
            \end{cases}
        \]e inoltre $ \bm{u}(0)=\bm{p}. $\qed
    \end{itemize}
}
%% END
\esempio{
    \[
        \begin{cases}
            x'=-y^{2}\\ 
            y'=x^{2}
        \end{cases}
    \]
    \begin{enumerate}
        \item \emph{Equilibri}. \[
            \bm{f}(x,y)=(-y^{2},x^{2}) = (0,0)
        \] 
        
        $\implies$ ho un solo punto di equilibrio, $ \bm{0} $. 
        \item \emph{Punti con tangente orizzontale o verticale}.
        
        Il vettore tangente ad una soluzione in $ (x,y) $ è $ \bm{f}(x,y) $, dunque: \begin{itemize}
            \item tangente verticale: $ x'=0 $ $ \iff $ $ y^{2} =0$ 
            
            $\implies$ l'asse $ x $ viene intersecato verticalmente dalle orbite;
            \item tangente verticale: $ y'=0 $ $ \iff $ $ x^{2}=0 $ 
            
            $\implies$ l'asse $ y $ viene intersecato orizzontalmente dalle orbite.
        \end{itemize}

        In $ (x,0) $ il vettore tangente alla soluzione che passa per quel punto è $ (0,x^{2}) $, dunque tutte le orbite che passano per l'asse delle $ x $ sono percorse verso l'alto.

        In $ (0,y) $ il vettore tangente alla soluzione che passa per quel punto è $ (-y^{2}, 0 ) $, dunque tutte le orbite che passano per l'asse delle $ y $ sono percorse verso sinistra.
        \item \emph{Orbite non singolari}
        
        Consideriamo i punti in cui $ f_1(x,y)\neq 0 $, ovvero i punti con $ y\neq 0 $. Cerchiamo l'equazione delle orbite dei punti che non sono sull'asse $ x $. \[
            \begin{cases}
                \varphi'(x)=-\frac{x^{2}}{\varphi^{2}(x)}\\
                \varphi(p_1)=p_2\tag*{(PC)\textsubscript{$\varphi$}}
            \end{cases}
        \]che è a variabili separabili: \begin{align*}
            \varphi^{2}(x)\,\varphi'(x) & = -x^{2}\\[2ex] 
            \int \varphi^{2}(x)\,\varphi'(x)\,dx & = \int -x^{2}\,dx\\[2ex] 
            \int \varphi\,d\varphi & = -\int x^{2}\,dx\\[2ex]
            \frac{1}{3} \varphi^{3}(x) &= -\frac{1}{3}x^{3} + c
        \end{align*}

        Imponendo il passaggio per $ \bm{p} $, otteniamo che $ \displaystyle c= \frac{1}{3}\, p_2^{3}$. 
        
        Dunque \[
            \varphi(x)=\sqrt[3]{p_2^{3}-x^{3}}
        \]e le orbite sono quelle mostrate in figura \ref{fig:orbiteradiceterza}
        \begin{figure}
            \begin{center}
                \input{matlab_graph/grafico1.tex}
            \end{center}
            \caption{Orbite per il problema di Cauchy}\label{fig:orbiteradiceterza}
        \end{figure}
    \end{enumerate}
}
\paragrafo{Esempio}{%
\[
    \begin{cases}
        x'=y^{2}\\ 
        y'=-xy
    \end{cases}    
\]
\begin{enumerate}
    \item \emph{Equilibri}
    
    Tutti i punti $ (x,0) $ sono equilibri.
    \item \emph{Tangente orizzontale e verticale}
    
    Tutta l'asse delle $ y $ è composta da punti a tangente orizzontale. 
    \item \emph{Orbite non singolari}
    
    Consideriamo $ f_1(x,y)\neq 0 $, ovvero quelli per i quali $ y\neq 0 $: troviamo tutte le orbite, infatti i punti $ (x,0) $ sono equilibri. Calcolando: \[
        \varphi'(x)=-\frac{x\,\varphi(x)}{\varphi^{2}(x)}=-\frac{x}{\varphi(x)}
    \]Da qui, possiamo dire \[
        \varphi(x)\,\varphi'(x)=-x \quad \implies \quad \varphi^{2}(x)=-x^{2}+c
    \]
    
    $c>0$, e $ \displaystyle \varphi(x)=\pm (c-x^{2})^{1/2}$. 

    Inoltre $ x'=y^{2}>0 $ lungo le orbite. Dunque il diagramma di fase è quello illustrato in figure \ref{fig:eqwsdcs}. 
    \begin{figure}
        \begin{center}
            \begin{tikzpicture}
                \draw [-Latex] (-3,0) -- (3,0);
                \draw [-Latex] (0,-3) -- (0,3);
                \draw [ultra thick] (-1.8,0) -- (1.8,0);
                \draw [ultra thick, dashed] (-1.8,0) -- (-2.9,0);
                \draw [ultra thick, dashed] (1.8,0) -- (2.9,0);
                \foreach \r in{0.5,1,...,2.5}{
                    \draw (0,0) circle (\r);
                    \fill [white] (\r,0) circle (0.05);
                    \draw (\r,0) circle (0.055);
                    \fill [white] (-\r,0) circle (0.05);
                    \draw (-\r,0) circle (0.055);
                    \draw [-Latex] (0,\r) -- (0.1, \r);
                    \draw [-Latex] (0,-\r) -- (0.1, -\r);
                };
            \end{tikzpicture}
        \end{center}
        \caption{Diagramma di fase per l'esempio \framref{jjdjdjdkjnsjdkjncskjndsckjn}}\label{fig:eqwsdcs}
    \end{figure}
\end{enumerate}
}{jjdjdjdkjnsjdkjncskjndsckjn}{}
